\documentclass[onecolumn, draftclsnofoot,10pt, compsoc]{IEEEtran}
\usepackage{graphicx}
\usepackage{url}
\usepackage{setspace}
\usepackage{cite}
\usepackage{geometry}
\usepackage{pgfgantt}
\geometry{textheight=9.5in, textwidth=7in}
% 1. Fill in these details
\def \CapstoneTeamName{                 The Ingineers}
\def \CapstoneTeamNumber{               22}
\def \GroupMemberOne{                   Brett Case}
\def \GroupMemberTwo{                   Alexander Morefield}
\def \GroupMemberThree{                 James Zeng}
\def \GroupMemberFour{                  Christopher Jansen}
\def \GroupMemberFive{                  Burton Jaursch}
\def \CapstoneProjectName{              Tesla Web Application}
\def \CapstoneSponsorCompany{           Ingineerix, Inc}
\def \CapstoneSponsorPerson{            Phil Sadow}

% 2. Uncomment the appropriate line below so that the document type works
\def \DocType{                  %Problem Statement
                                %Requirements Document
                                %Technology Review
                                Design Document
                                %Progress Report
                                }

\newcommand{\NameSigPair}[1]{\par
\makebox[2.75in][r]{#1} \hfil   \makebox[3.25in]{\makebox[2.25in]{\hrulefill} \hfill            \makebox[.75in]{\hrulefill}}
\par\vspace{-12pt} \textit{\tiny\noindent
\makebox[2.75in]{} \hfil                \makebox[3.25in]{\makebox[2.25in][r]{Signature} \hfill  \makebox[.75in][r]{Date}}}}
% 3. If the document is not to be signed, uncomment the RENEWcommand below
\renewcommand{\NameSigPair}[1]{#1}

%%%%%%%%%%%%%%%%%%%%%%%%%%%%%%%%%%%%%%%
\begin{document}
\begin{titlepage}
    \pagenumbering{gobble}
    \begin{singlespace}
        % \includegraphics[height=4cm]{coe_v_spot1}
        \hfill
        % 4. If you have a logo, use this includegraphics command to put it on the coversheet.
        %\includegraphics[height=4cm]{CompanyLogo}
        \par\vspace{.2in}
        \centering
        \scshape{
            \huge CS Capstone \DocType \par
            {\large\today}\par
            \vspace{.5in}
            \textbf{\Huge\CapstoneProjectName}\par
            \vfill

▽
            \vfill
            {\large Prepared for}\par
            \Huge \CapstoneSponsorCompany\par
            \vspace{5pt}
            {\Large\NameSigPair{\CapstoneSponsorPerson}\par}
            {\large Prepared by }\par
            % Group\CapstoneTeamNumber\par
            % 5. comment out the line below this one if you do not wish to name your team
            % \CapstoneTeamName\par
            \vspace{5pt}
            {\Large
                \NameSigPair{\GroupMemberOne}\par
                \NameSigPair{\GroupMemberTwo}\par
                \NameSigPair{\GroupMemberThree}\par
                \NameSigPair{\GroupMemberFour}\par
                \NameSigPair{\GroupMemberFive}\par
            }
            \vspace{20pt}
        }
        \begin{abstract}
        % 6. Fill in your abstract
        The purpose of this document is to discuss and outline the architecture and design choices made for the Tesla Web Application
        The functionality and design choices made for each of the technologies is discussed in detail.
        We will go into what each piece of our application will do and how it will interact with others.
        \end{abstract}
    \end{singlespace}
\end{titlepage}
\newpage
\pagenumbering{arabic}
\tableofcontents
% 7. uncomment this (if applicable). Consider adding a page break.
%\listoffigures
%\listoftables
\clearpage

% 8. now you write!

\section{Overview}

    \subsection{Scope}
        This document describes the technologies we will use to build a web application to control Tesla cars. This document serves as a high-level overview of the process and will not include very small details.
    \subsection{Purpose}
        This document specifies the technologies that will be used to build a Tesla Web Application. This information will be followed through the development process.
        
    \subsection{Intended Audience}
        This document is intended for any person working to build our Tesla Web Application. It lays out all of the technologies that will be used and how they will interact with each other. 

\section{Definitions}
\begin{itemize}
    \item \textbf{3.1 ReactJS:} A component based JavaScript Library for building user interfaces. 
    \item \textbf{3.1 Header:} The element at the top of the web app containing the navigation bar.
    \item \textbf{3.2 Style Sheets:} The template that contains the style of the page independent from the content.
    \item \textbf{3.2 Sass (Syntactically Awesome Style Sheets):} A scripting language that is compiled into CSS.
    \item \textbf{3.2 CSS (Cascading Style Sheets):} A style sheet language used for describing the style of a document.
    \item \textbf{3.3 Electron:} An open source library developed for building cross-platform desktop applications with HTML, CSS, and JavaScript.
    \item \textbf{3.3 HTML (Hyper Text Markup Language):} The markup language used to create web pages.
    \item \textbf{3.3 Chromium:} An open-source web browser.
    \item \textbf{3.3 Node.js:} An open-source JavaScript runtime environment that executes javaScript code outside of a browser.
    \item \textbf{3.3 Runtime:} A program library used to implement functions.
    \item \textbf{3.4 MySQL:} An open-source relational database management system.
    \item \textbf{3.5 Component:} Independent reusable pieces that describe what should be shown on screen.
    \item \textbf{5.1 HTTPS (Hyper Text Transfer Protocol Secure):} A secure framework that transfers information between a browser and a site. 
    \item \textbf{5.2 Web Application Firewall:} Enforces rules on a website.
    \item \textbf{7.2 DBMS (database management system):} System software that provides programmers the ability to create databases, and a systematic way to create, retrieve, update, and manage data. 
    \item \textbf{7.2.3 Entity:} An object in the database about which data is stored.
\end{itemize}
    

\section{Front-End}
    \subsection{Structure}
        The front end of this web application will be built up using ReactJS components.
        Each component will represent a specific part of the user interface that when combined, will create the entire page.
        An example of a component would be the header of the application or the side menu.
        The advantage to having each piece of the application being its own component is it will allow us to easily change or swap out a piece as needed.
    \subsection{Styling}
        The styling of the page will be done with Sass instead of regular CSS.
        Sass gets compiled turning it into CSS meaning that before the web server is booted up, there will need to be an extra step of to turn Sass into CSS.
        Using Sass will let us use variables in our style sheets to make changing any colors or values easy.
        \subsubsection{Web Application}
            The web application will utilize a dynamic page along with a navigation header to provide users with a familiar environment that can scale to match screen real-estate. 
        \subsubsection{Mobile View}
            Using the web application on a mobile device should re-style the page to utilize a style that closely matches the official Tesla application. This will allow us to provide users with a familiar experience.
    \subsection{Application Framework}
        The framework used for the native desktop application will be electron. Electron allows us to HTML, Sass compiled CSS, and ReactJS to build native desktop applications by combining Chromium and Node.js runtimes into an application. This provides us the opportunity to build a native desktop application based upon the same technologies our web application uses.
    \subsection{Integration with Back-End}
        All the data will be stored in a MySQL database that will be polled to get the content for the front-end pages.
    \subsection{Testing}
        Since the page will be made up of many components, testing will involve targeting a specific component and making sure that all data is being properly grabbed from the database. Further testing will involve pressing the buttons that will control the car and make sure they perform the correct action.
    \subsection{Timeline}
        There will be 4 weeks spent on the Research and design portion of the front end.
        Following that will be a 3 week period of building the front-end to match our design without reusable components yet.
        After that there will be 3 weeks of breaking the site down into ReactJS components to have the most reusability.
        We will now be at the end of Winter term and the front-end should be finished.
        Spring term will be about testing our front-end to make sure that it is working as expected and that it is properly getting data from the database.
        If there are any small features that need to be added later, they will be added during this time.
        We will cut off addition of any new features 3 weeks before the Engineering Expo to ensure that what is currently in the application has been sufficiently tested and up to standards.

\section{Back-End}
    \subsection{Purpose}
        Back-end frameworks are the collection of tools and languages used in server-side programming. This is the core logic that the application uses and especially the information that is given and taken by an application \cite{tripathi_2018}. Within the Tesla Web Application, this will generally be the agent that will communicate with the Tesla API to the car and front-end. It will be important for this application to relay information regarding the car quickly and easily for the front-end to render the information for the user. 
    \subsection{Structure and Expectations}
        The back-end framework will be running on the server to collect requests from individual clients and process their requests. It is important for the server to be able to route the requests to the right areas. Also, the framework should be capable of sufficiently scaling to avoid having the system be rebuilt like the original version of this application being replaced by this project.
        
        Considering this application will be web-based, it is almost entirely essential for the back-end framework to have REST functionality. As a bonus, the application should have decent support and possibly documentation with being able to work with the React front-end and MySQL database. 
    \subsection{Languages}
        While deciding on the framework is generally the most interesting part of creating the back-end, it could be considered even more important to consider the language that will be used for the back-end and what the framework will be in. The language will generally provide what to expect from the frameworks that come from them, whether that could be shortcomings or perks.
        
        For this project, we are moving forward with implementing python as our back-end language. Python is designed to be an incredibly portable, yet secure language. This will help ensure that users of the application will be safer when using the application, and opens the door for the possibility of a native application that uses this back-end. Further, the language generally runs pretty fast, faster than other possibilities that we considered like PHP. Finally, one of the biggest strengths for python is the dedicated community and developers who provide lots of documentation and guides for the language\cite{intersog_2017}. While our group has experience working with python within the OSU computer science curriculum, it will also help to brush up and go deeper into the language than we have gone with simple scripting. 
    \subsection{Frameworks}
        Because we've chose python as the best suited language for the job, it was quickly decided that Django would be the preferred back-end framework. Django is beloved by those who use it, and has a great community and documentation for learning the language and figuring out bugs. Like React, Django also uses components, which are highly portable and reusable to avoid code duplication.\cite{malhotra_2018}
        
        As mentioned, many of the perks of Django are rooted in the perks of python, but within the general scope of the project, Django already has decent documentation working with React as well as MySQL. This will make developing much easier with multiple guides on how to set up Django with both of these technologies
    \subsection{Testing}
        The main testing that will need to be done with Django is ensuring seamless integration with MySQL and being able to use the Tesla API. While we are not certain how this will look, it will be important in the early stages of developing the back-end that the it can use the API to communicate with the car, can complete these requests quickly and securely.
        
        Security will also be another issue where we will need to test that the back-end and effectively authenticate a client with the car they are attempting to control. We will need to make sure that the back-end can work with the API specifically for this issue so those who try to wrongly access one of the cars this application will be used for cannot cause the car to act improperly. 
    \subsection{Timeline}
        The development schedule of the back-end will be tied closely with the development of the database. We assume that the development of the back-end will take roughly 5-6 weeks spaced in between the understanding of the Tesla API and creation of the database. The front-end development will take place during part of this time, though most of the communication with the back-end will happen near the end of the back-end development. In the context of the year, we hope to have the back-end completed well before the end of winter term so focus can be placed on the front-end design.  
\section{Security}
    \subsection{Purpose}
    Given that the application gives the user the ability to start and drive the vehicle, it is important that information is stored and transferred in a secure manner.  
    \subsection{HTTPS}
    Like HTTP, HTTPS is a framework that transfers information back and froth between a browser and a site. HTTPS encrypts the data being sent at the socket security layer (SSL) or the transport security layer (TLS). The information is encrypted using a asymmetric public key. When encrypting and decrypting data, there is a private and public key. The public key is held by everyone, while the private key is held by a specific user.\cite{HTTPS} 
    \subsection{Web Application Firewall}
    A Web Application Firewall (WAF) checks both the get and post functions of a website. Then based on specifications made by the site it will flag suspicious activity. This can range from suspicious URL manipulation to cross site scripting and SQL injection attempts. If something is detected the site will present the user with a CAPTCHA. If the CAPTCHA was not entered correctly the site will not process any information requested by the user. The security levels are able to be adjusted in the WAF.\cite{WAF}
\section{Talking to the Car}
    \subsection{API}
    In order to provide the basic functions of our application we will be using the Tesla API. Tesla's API as documented by Tim Dorr will allow our application to lock and unlock the car, start the car, heat and cool the car, etc. We will also be able to see the status of the car and other vehicle information. To access this information and interact with the API, we will be sending JSON objects over a VPN. \cite{API}
\section{Database}

    \subsection{Purpose}
    
        A database will be used to store information about the cars that can be controlled using the application and store other records needed for the website.
        
    \subsection{Structure}
    
        To do this, we will be using MySQL as a database management system.
        MySQL stores data in tables, and allows users to define relationships between and rules about data fields in these tables\cite{e2}.
        The primary tables we'll be making are the Car and User tables, however there are some miscellaneous tables we'll need to create to represent things such as relationships and vehicle diagnostic information.
        
        \subsubsection{Car Table}
        
            This table will include basic info about the car, such as the model, but also dynamically updated information such as the charge, estimated range, location, whether it's locked or unlocked.
            It will also need a relation to a user, and a relation to diagnostic snapshots.
            
        \subsubsection{User Table}
        
            This table will include information about users, including their user names, hashed passwords, and a relation to at least one vehicle.
            
        \subsubsection{Relationship Tables}
        
            We will need to make tables to represent the various 1:N relationships we will have between entities in the database.
            For instance, we will need relate the user User and Car table, the Car table with its record of diagnostic snapshots.
            
    \subsection{Timeline}
    
        The database structure will be put together during the first week of development, as it needs to be in place to begin development of the back-end.
        We will need to populate the database with the existing users of our client's cars, but otherwise the contents of the database will slowly be populated by the back-end through use the site.

\section{Hosting}

    Since our project is going to be web-based, our project will need to be contained on a web server that our users can access.
    We will be using our our client's servers for this, which at this point is the most convenient and secure option. While there may be possibilities to host the application on Oregon State servers during development, the security for the client is prioritized. However, this would help speed of development as our group is already familiar with connecting to Oregon State servers. It will still need to be explored how the application can be tested using the Tesla API, whether this would be with an actual vehicle, or a simulated one. 

\bibliographystyle{IEEEtran}
\bibliography{references}
\end{document}