\documentclass[onecolumn, draftclsnofoot,10pt, compsoc]{IEEEtran}
\usepackage{graphicx}
\usepackage{url}
\usepackage{setspace}
\usepackage{cite}
\usepackage{geometry}
\usepackage{longtable}
\usepackage{pgfgantt}
\geometry{textheight=9.5in, textwidth=7in}
% 1. Fill in these details
\def \CapstoneTeamName{                 The Ingineers}
\def \CapstoneTeamNumber{               22}
\def \GroupMemberOne{                   Brett Case}
\def \GroupMemberTwo{                   Alexander Morefield}
\def \GroupMemberThree{                 James Zeng}
\def \GroupMemberFour{                  Christopher Jansen}
\def \GroupMemberFive{                  Burton Jaursch}
\def \CapstoneProjectName{              Tesla Web Application}
\def \CapstoneSponsorCompany{           Ingineerix, Inc}
\def \CapstoneSponsorPerson{            Phil Sadow}

% 2. Uncomment the appropriate line below so that the document type works
\def \DocType{                  %Problem Statement
                                %Requirements Document
                                %Technology Review
                                %Design Document
                                Progress Report
                                }

\newcommand{\NameSigPair}[1]{\par
\makebox[2.75in][r]{#1} \hfil   \makebox[3.25in]{\makebox[2.25in]{\hrulefill} \hfill            \makebox[.75in]{\hrulefill}}
\par\vspace{-12pt} \textit{\tiny\noindent
\makebox[2.75in]{} \hfil                \makebox[3.25in]{\makebox[2.25in][r]{Signature} \hfill  \makebox[.75in][r]{Date}}}}
% 3. If the document is not to be signed, uncomment the RENEWcommand below
\renewcommand{\NameSigPair}[1]{#1}

%%%%%%%%%%%%%%%%%%%%%%%%%%%%%%%%%%%%%%%
\begin{document}
\begin{titlepage}
    \pagenumbering{gobble}
    \begin{singlespace}
        % \includegraphics[height=4cm]{coe_v_spot1}
        \hfill
        % 4. If you have a logo, use this includegraphics command to put it on the coversheet.
        %\includegraphics[height=4cm]{CompanyLogo}
        \par\vspace{.2in}
        \centering
        \scshape{
            \huge CS Capstone \DocType \par
            {\large\today}\par
            \vspace{.5in}
            \textbf{\Huge\CapstoneProjectName}\par
            \vfill

▽
            \vfill
            {\large Prepared for}\par
            \Huge \CapstoneSponsorCompany\par
            \vspace{5pt}
            {\Large\NameSigPair{\CapstoneSponsorPerson}\par}
            {\large Prepared by }\par
            % Group\CapstoneTeamNumber\par
            % 5. comment out the line below this one if you do not wish to name your team
            % \CapstoneTeamName\par
            \vspace{5pt}
            {\Large
                \NameSigPair{\GroupMemberOne}\par
                \NameSigPair{\GroupMemberTwo}\par
                \NameSigPair{\GroupMemberThree}\par
                \NameSigPair{\GroupMemberFour}\par
                \NameSigPair{\GroupMemberFive}\par
            }
            \vspace{20pt}
        }
        \begin{abstract}
        % 6. Fill in your abstract
        This document will cover the progress our group has made over the past 10 weeks for the Tesla Web Application. Inside will be our problems and solutions, a brief description of the problem, and a weekly breakdown of what was happening in the project.
        \end{abstract}
    \end{singlespace}
\end{titlepage}
%\newpage
%\pagenumbering{arabic}
\tableofcontents
% 7. uncomment this (if applicable). Consider adding a page break.
%\listoffigures
%\listoftables
\clearpage

% 8. now you write!
\section{Introduction}
        Over the course of this year, our group will be working on a web application, designed for mobile phones, that provides functionality for blacklisted Tesla vehicles. This application is designed for our client, Phil Sadow, who works on these types of vehicles. It should be noted that this application is actually a replacement for a previous application which Sadow designed for his clients. This design, while functional, does not meet usability or scalability standards that most modern web applications meet. Over the past 10 weeks, we have developed plans on how this replacement web application will be created, and will start development in early 2019. 
        
        This document briefly explains the design of the application so far, and the process we have taken along the way. This includes the variety of problems we have run into, a proposed timeline of our future development, and the weekly breakdown of our progress completing the design of this system. While this system's design is not entirely completed, this progress report shows our current design choices, which are open for change as we continue iterating through our design and development process. 
\section{Project Information}
    \subsection{Purpose}
        When a Tesla is totaled, Tesla is known for blacklisting the vehicle's VIN. The owner of this vehicle is prevented from using the official Tesla app, and getting spare parts. Our project attempts to give these grey market Teslas the functionality of Tesla's native app. Grey market Tesla owners will again be able to lock/unlock their car, start their engine, control their temperature, and more.
    \subsection{Goals}
        The goals of this assignment are simple. We want to mimic the Tesla app to the best of our ability. We want to have all of the key features the Tesla app has, such as locking and unlocking their car, starting the engine, managing the temperature, and summoning the vehicle. With access to the Tesla API and full diagnostics, we can improve of the functionality of Tesla's current application, and provide functionality shown by Phil's previous iteration of this application. We also want the new design of the application to be more professional and modern, while maintaining security to prevent people from gaining unauthorized access to the vehicle.  
\section{Where we are now}
    \subsection{Design}
        \subsubsection{Front-end}
            The front end will be built up of ReactJS components to let pieces be modular and reusable. ReactJS also keeps a virtual DOM of the page making it speedy to render what is needed. The styling will be done in Sass to give us an easier way to make easy changes across the document thanks to the ability to use variables.
        \subsubsection{Back-end}
            We will be creating our back-end using Django, which is a python-based MVC framework. The back-end will be sitting on the server taking in responses and routing them to the correct places. Django is very quick to develop in, and does a great job with creating a good amount of customization within the framework. The back-end will also be forwarding requests from the application to the Tesla vehicles, so it is important for the framework to quickly and accurately authenticate legitimate users and reject bad requests.  
        \subsubsection{Security}
            For security we decided to use HTTPS to encrypt data as it is being transmitted between the site and the users browser. We will also use a web application firewall to prevent suspicious activity on our site. This will prevent things like cross site scripting and SQL injection.
        \subsubsection{API}
            The API is used to communicate with the vehicle. We will sill be using the Tesla API as documented by Tim Dorr. The API allows us to provide the functionality of the app. In order to communicate with the API, we will be sending it JSON objects through a VPN connection. The API also provides the ability to incorporate fleets into our application. Currently Phil Sadow's version of the application each account can only have one vehicle. This means that if a user has more than one vehicle they must make a new account for every additional vehicle they own. The Tesla API allows us to sort and look at different vehicle information based on its id and vehicle id. Another useful feature the API provides is access to the Vehicle's GUI information. From here we can see how much battery is left on the car and how far it can go. If we decide to do a GUI for our application, we will be able to use this information to display information to our users.
        \subsubsection{Database}
            For the database, we have decided to use MySQL.
            The idea now is to have tables to represent the users, their cars, and the car's diagnostic information, as well as the relationships between them.
    \subsection{Time line}
    %---Gantt Chart Goes Here---%
    \begin{ganttchart}[vgrid,
     vrule/.style={very thick, blue}]{1}{30}
    \gantttitle{Tesla Web Application}{30} \\
    \gantttitle{Fall Term}{10}
    \gantttitle{Winter Term}{10}
    \gantttitle{Spring Term}{10} \\
    \gantttitlelist{1,...,30}{1} \\
    \ganttbar{Design}{1}{10} \\
    \ganttbar{Re-write Backend}{11}{13} \\
    \ganttlinkedbar{API Integration}{14}{16} \\
    \ganttlinkedbar{Backend Diagnostics}{17}{20} \\
    \ganttlinkedbar{API Controls Car}{21}{25} \\
    \ganttbar{UX Research}{11}{14} \\
    \ganttlinkedbar{UX Design}{15}{16} \\
    \ganttlinkedbar{Front End Template}{16}{17} \\ 
    \ganttlinkedbar{Componetize Front End}{18}{20} \\
    \ganttbar{Testing}{23}{30} \\
    \ganttbar{Design Booth}{27}{28} \\
    \ganttlinkedbar{Create Booth}{29}{30}
    \ganttlink{elem4}{elem9}
    \ganttlink{elem8}{elem9}
    \end{ganttchart}
        
\section{Problems Encountered}
    \subsection{Problems}
    Working in new group is always difficult at first as you figure out what everyone else can do. Luckily this is normally gone within a week or two of the group forming. We had a bigger issue start to emerge over the course of the term however. As time went on our client got more and more unresponsive, eventually not responding at all. This lead us to have to make our own assumptions on how the design should be so we could turn in our documents on time. 
    \subsection{Solution}
    To try and get a response from our client we tried to break things down into smaller messages to not bombard him with a wall of text to read. We also asked to set up weekly meetings to have a predetermined time we knew we could ask questions. We tried spacing out our messages to give the client time to focus on one before giving him another. 
    \newpage
\section{Retrospective}

    \begin{longtable}{p{.07\textwidth} | p{.275\textwidth} | p{.275\textwidth} | p{.275\textwidth}}
    
        \textbf{ }
        & \textbf{Positives}
        & \textbf{Problems}
        & \textbf{Actions}
        \\\hline
        Week 3
        &   \begin{itemize}
                \item Set up communication with each other and our client.
                \item Met with our client and got enough information for Problem Statement.
                \item Completed individual drafts of Problem Statement.
            \end{itemize}
        &
        &
        \\Week 4
        &   \begin{itemize}
                \item Completed Group Problem Statement.
                \item Devised plan to complete Requirements Document.
            \end{itemize}
        &   \begin{itemize}
                \item Slow initial progress on Requirement Document.
            \end{itemize}
        &
        \\Week 5
        &   \begin{itemize}
                \item Worked on Requirements Document.
                \item Got some more information from our client.
            \end{itemize}
        &   \begin{itemize}
                \item Waited too long to send questions to our client, would have been to late if the deadline wasn't extended.
            \end{itemize}
        &   \begin{itemize}
                \item We tried to be more proactive with our communications.
            \end{itemize}
        \\Week 6
        &   \begin{itemize}
                \item Submitted our Requirements Document.
                \item Brainstormed and assigned topics for Tech Reviews.
                \item Completed Team Standards document.
            \end{itemize}
        & \begin{itemize}
            \item Lots of people in group so some topics were a little vaporous.
        \end{itemize}
        &  
        \\Week 7
        &   \begin{itemize}
                \item Completed Tech Reviews.
            \end{itemize}
        &   \begin{itemize}
                \item Didn't review or look over group members tech reviews
                \item Had not followed up with client since initial meeting
            \end{itemize}
        &
        \\Week 8
        &   \begin{itemize}
                \item Sent information over to client
            \end{itemize}
        &   \begin{itemize}
                \item Client became unresponsive, we have no feedback.
                \item We had no github repository since our client wanted to be the owner.
                \item Had issue meeting with TA to review work so far
            \end{itemize}
        &   \begin{itemize}
                \item We spoke with the professors about our problems.
                \item We set up a repo that just contains documentation for the class.
            \end{itemize}
        \\Week 9
        &   \begin{itemize}
                \item Started to plan design document
            \end{itemize}
        &   \begin{itemize}
                \item Still no reply from client.
            \end{itemize}
        &
        \\Week 10
        &   \begin{itemize}
                \item Submitted Design Document.
                \item Devised a plan for our Progress Report.
                \item Met up with our TA
            \end{itemize}
        &   \begin{itemize}
                \item Client still missing.
            \end{itemize}
        &   \begin{itemize}
                \item We sent him one last email with all of our documents asking he please review them before December 3rd.
            \end{itemize}
    
    \end{longtable}

\bibliographystyle{IEEEtran}
\bibliography{references}
\end{document}

