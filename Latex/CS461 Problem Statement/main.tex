\documentclass[onecolumn, draftclsnofoot,10pt, compsoc]{IEEEtran}
\usepackage{graphicx}
\usepackage{url}
\usepackage{setspace}
\usepackage{cite}
\usepackage{geometry}
\geometry{textheight=9.5in, textwidth=7in}
% 1. Fill in these details
\def \CapstoneTeamName{                 The Cleverly Named Team}
\def \CapstoneTeamNumber{               22}
\def \GroupMemberOne{                   Brett Case}
\def \GroupMemberTwo{                   Alexander Morefield}
\def \GroupMemberThree{                 James Zeng}
\def \GroupMemberFour{                  Christopher Jansen}
\def \GroupMemberFive{                  Burton Jaursch}
\def \CapstoneProjectName{              Tesla Web Application}
\def \CapstoneSponsorCompany{           Ingineerix, Inc}
\def \CapstoneSponsorPerson{            Phil Sadow}

% 2. Uncomment the appropriate line below so that the document type works
\def \DocType{          Problem Statement
                                %Requirements Document
                                %Technology Review
                                %Design Document
                                %Progress Report
                                }

\newcommand{\NameSigPair}[1]{\par
\makebox[2.75in][r]{#1} \hfil   \makebox[3.25in]{\makebox[2.25in]{\hrulefill} \hfill            \makebox[.75in]{\hrulefill}}
\par\vspace{-12pt} \textit{\tiny\noindent
\makebox[2.75in]{} \hfil                \makebox[3.25in]{\makebox[2.25in][r]{Signature} \hfill  \makebox[.75in][r]{Date}}}}
% 3. If the document is not to be signed, uncomment the RENEWcommand below
\renewcommand{\NameSigPair}[1]{#1}

%%%%%%%%%%%%%%%%%%%%%%%%%%%%%%%%%%%%%%%
\begin{document}
\begin{titlepage}
    \pagenumbering{gobble}
    \begin{singlespace}
        % \includegraphics[height=4cm]{coe_v_spot1}
        \hfill
        % 4. If you have a logo, use this includegraphics command to put it on the coversheet.
        %\includegraphics[height=4cm]{CompanyLogo}
        \par\vspace{.2in}
        \centering
        \scshape{
            \huge CS Capstone \DocType \par
            {\large\today}\par
            \vspace{.5in}
            \textbf{\Huge\CapstoneProjectName}\par
            \vfill

▽
            \vfill
            {\large Prepared for}\par
            \Huge \CapstoneSponsorCompany\par
            \vspace{5pt}
            {\Large\NameSigPair{\CapstoneSponsorPerson}\par}
            {\large Prepared by }\par
            % Group\CapstoneTeamNumber\par
            % 5. comment out the line below this one if you do not wish to name your team
            % \CapstoneTeamName\par
            \vspace{5pt}
            {\Large
                \NameSigPair{\GroupMemberOne}\par
                \NameSigPair{\GroupMemberTwo}\par
                \NameSigPair{\GroupMemberThree}\par
                \NameSigPair{\GroupMemberFour}\par
                \NameSigPair{\GroupMemberFive}\par
            }
            \vspace{20pt}
        }
        \begin{abstract}
        % 6. Fill in your abstract
        One thing that makes Teslas unique is that Tesla, the company, doesn't want their cars to be repaired after they have been totaled.
        Because of this, Teslas that hold a salvage title do not get all the support that a new Tesla would get, including access to the Tesla app, which gives the user a control panel for their cars they can use from their phones.
        There has been some work done on an existing web application that can be used on these repaired Teslas.
        This has been initiated by Ingineerix Inc., however, it requires an overhaul to the back-end to control the car through the vehicle API, and enhancements to the application to improve the user experience and usability.
        \end{abstract}
    \end{singlespace}
\end{titlepage}
\newpage
\pagenumbering{arabic}
\tableofcontents
% 7. uncomment this (if applicable). Consider adding a page break.
%\listoffigures
%\listoftables
\clearpage

% 8. now you write!
\section{Definition and Description}
To maintain the quality and reputation of their cars, Tesla has a policy to stop direct support of vehicles after they have been totaled \cite{1}.
While most companies provide full details and instructions to repair and modify their cars, Tesla refuses to accept the status quo due to the sharp backlash from the media whenever their cars are involved in an accident, even in events when the driver or vehicle are not liable. 
The fear for Tesla is that if one of their totaled vehicles is repaired by a 3rd-party and becomes involved in accident due to the fault of the technician, the brand would still come under fire.
While the reasoning for this decision is understandable, there is still a barrier created between competent mechanics and cutting-edge vehicles.
Further the mechanics are not taking advantage of these damaged Teslas because of how hard Tesla has made it to get parts and learn how to work on them. 
This results in many of these salvage Teslas that could be back out on the roads, but instead are rusting away.
\\ 
There is some hope, however, as these broken cars can not only be repaired, but done so for cheap since many people do not want to deal with the overhead of trying to get service manuals and spare parts.
One of those up to the job is Phil Sadow of Ingineerix, Inc., who has worked on over 400 totaled Tesla vehicles, including the Model S, Model X, and the new Model 3.
Keeping with Tesla’s theme of not supporting the repair of their cars, their first-party app to control the car no longer works after a car has been totaled.\cite{1}
The features within the app can be starting the car, A/C, audio system, and even “summoning” the vehicle where it will autonomously drive to your location.
Any person who has bought one of these repaired cars will not have access to the features provided by the app because the car will be put onto a blacklist following a large crash, preventing the owner from using the app on the car or getting replacement parts.\cite{1} 
This project exists to fill the hole that Tesla has left open: create a way for owners of these repaired Tesla cars to use the features provided by the first-party app.
By making this application, the people how have these repaired Teslas will be able to take advantage of all the features their car has again.
\section{Problem Solution}
There is already some of the application in place but it still requires a bit of work.
In it's current state, the application is a self-phrased “bare-bones web app” which can re-establish the connection between the owner and automobile. 
From the point-of-view of the car, it believes it is still talking to the Tesla servers. 
This web application is simply a static web page so it can work on any mobile device with a browser.
The current implementation of the back-end of this system is essentially a Perl script with other background support scripts, which Phil has described as ``... horrible. It’s basically duct tape and coat hangers".
\\
The goal then of this project is to completely overhaul both the front-end and back-end of the system, provide structure, incorporate the latest web technologies, and provide a smooth, clean experience for owners.
The front end development will require creating an attractive user interface and using client side javascript using AJAX calls to make the page dynamic.
There is currently a front end in place with the bare minimum but it will require some extensive work to update it to be more user friendly.
Some technologies that will be used are JSON, Ajax, WS, and using a REST architecture.
The back end development will need to be able to access a database that has all of the cars in it and connect to a car through an API that has been heavily documented.\cite{2}
The data will be sent through an openVPN connection to the car and all data will be encrypted with HTTPS. 
The current backend has all of the functionality in place but due to the lack of design and structure, needs to be rewritten in a more modern language.
This means that the logical flow of how the program works can be kept mostly intact but will be brought into current standards.

The user will interact with the car through a webpage.
One of the focuses within the front-end development, is the concept of USer Experience, or how the user feels, interacts, and experiences an application.
One of the interests will be the clients who work on a variety of cars at a body repair shop, who consistently do business with Phil. 
Currently there is no easy way to switch between cars that you’re using since there is only one login per car.
This means that if a person owns multiple of these Teslas, it could be a bad user experience as they would have to constantly log in and out of different accounts for each car.
The plan is to add in a way to manage ``fleets". The user will be able to have a list of which cars are theirs and then select the vehicle they want to control from the app.
The requires users to be able to register multiple cars to their name. In order to do this the database will need to be updated so one user can have multiple cars.

Security is going to be an issue because as of right now if someone gets a hold of the system, they can locate a car, unlock it, and start it.
This is obviously a security issue and can be fixed by requiring a more robust log in process and possibly another way of in person authenticating to confirm the person trying to access the car is actually the owner.

The API needs to get into the cars diagnostics and show the user what is going on inside the car and how they can control it.
This will be accomplished through Ajax calls to the API to update the front end for the user.
This way all information about the car will be current.

\section{Performance Metrics}
The good news with this project is that there is an example to shoot for: Tesla's own first-party app.
Our web app should look similar to the Tesla app and should have all of the same features not limited to unlocking and locking the car, turning the car on, adjusting climate controls, and summoning the car to the user’s location.
While this is a hard metric to shoot for, this project can go beyond the functionality of the Tesla App.
With Phil Sadow's current application, the diagnostic capabilities are more open and available to his clients than what is given by Tesla.

The front end should be easy to navigate and professional looking.
In order to judge this, we will need to conduct user interviews and walk-throughs that follow good practices of UX design. 
The car actions should be prominently displayed and the option to switch cars should be available and simple to use.
The page needs to update dynamically in order to show current diagnostic information about the car.
All diagnostic information should be cleanly displayed with main features (charge level, temperature) available on the main page for control and more maintenance type features on a separate page to assist repairs.

The back end of the app should hold a database of all the cars while keeping all this data secure.
Any API calls should be encrypted and send properly to the car causing the correct action.
API calls should also occur in a reasonable time to provide the users with a quick, responsive experience.
There should be the ability for a user to have a ``fleet" of Teslas that they own that they can switch which one they are controlling easily to fit Sadow's clients.

\bibliographystyle{IEEEtran}
\bibliography{references}
\end{document}