\documentclass[onecolumn, draftclsnofoot,10pt, compsoc]{IEEEtran}
\usepackage{graphicx}
\usepackage{url}
\usepackage{setspace}
\usepackage{cite}
\usepackage{geometry}
\usepackage{pgfgantt}
\geometry{textheight=9.5in, textwidth=7in}
% 1. Fill in these details
\def \CapstoneTeamName{                 The Ingineers}
\def \CapstoneTeamNumber{               22}
\def \GroupMemberOne{                   Brett Case}
\def \GroupMemberTwo{                   Alexander Morefield}
\def \GroupMemberThree{                 James Zeng}
\def \GroupMemberFour{                  Christopher Jansen}
\def \GroupMemberFive{                  Burton Jaursch}
\def \CapstoneProjectName{              Tesla Web Application}
\def \CapstoneSponsorCompany{           Ingineerix, Inc}
\def \CapstoneSponsorPerson{            Phil Sadow}

% 2. Uncomment the appropriate line below so that the document type works
\def \DocType{                  %Problem Statement
                                Requirements Document
                                %Technology Review
                                %Design Document
                                %Progress Report
                                }

\newcommand{\NameSigPair}[1]{\par
\makebox[2.75in][r]{#1} \hfil   \makebox[3.25in]{\makebox[2.25in]{\hrulefill} \hfill            \makebox[.75in]{\hrulefill}}
\par\vspace{-12pt} \textit{\tiny\noindent
\makebox[2.75in]{} \hfil                \makebox[3.25in]{\makebox[2.25in][r]{Signature} \hfill  \makebox[.75in][r]{Date}}}}
% 3. If the document is not to be signed, uncomment the RENEWcommand below
\renewcommand{\NameSigPair}[1]{#1}

%%%%%%%%%%%%%%%%%%%%%%%%%%%%%%%%%%%%%%%
\begin{document}
\begin{titlepage}
    \pagenumbering{gobble}
    \begin{singlespace}
        % \includegraphics[height=4cm]{coe_v_spot1}
        \hfill
        % 4. If you have a logo, use this includegraphics command to put it on the coversheet.
        %\includegraphics[height=4cm]{CompanyLogo}
        \par\vspace{.2in}
        \centering
        \scshape{
            \huge CS Capstone \DocType \par
            {\large\today}\par
            \vspace{.5in}
            \textbf{\Huge\CapstoneProjectName}\par
            \vfill

▽
            \vfill
            {\large Prepared for}\par
            \Huge \CapstoneSponsorCompany\par
            \vspace{5pt}
            {\Large\NameSigPair{\CapstoneSponsorPerson}\par}
            {\large Prepared by }\par
            % Group\CapstoneTeamNumber\par
            % 5. comment out the line below this one if you do not wish to name your team
            % \CapstoneTeamName\par
            \vspace{5pt}
            {\Large
                \NameSigPair{\GroupMemberOne}\par
                \NameSigPair{\GroupMemberTwo}\par
                \NameSigPair{\GroupMemberThree}\par
                \NameSigPair{\GroupMemberFour}\par
                \NameSigPair{\GroupMemberFive}\par
            }
            \vspace{20pt}
        }
        \begin{abstract}
        % 6. Fill in your abstract
        This document will cover the requirements for the Tesla Web App capstone group. Inside are the specifications that we will follow as we implement our design.
        \end{abstract}
    \end{singlespace}
\end{titlepage}
\newpage
\pagenumbering{arabic}
\tableofcontents
% 7. uncomment this (if applicable). Consider adding a page break.
%\listoffigures
%\listoftables
\clearpage

% 8. now you write!
\section{Introduction}
This project exists to provide a way to control a Tesla car after it has been totaled.
Currently, if a Tesla car gets into a serious accident holds, it loses all support from Tesla including the use of their application.
Our application will fill the void and let salvage title Tesla car owners have a way to remotely control their car again.

\section{Definition, Acronyms, and Abbreviations}
\subsection{Definitions}
\begin{itemize}
    \item \textbf{3.1 User Interface:} The way a user would interact with a piece of software.
    \item \textbf{3.1 User Experience:} The overall experience the user has using a piece of software.
\end{itemize}
\subsection{Acronyms and Abbreviations}
\begin{itemize}
    \item \textbf{3.1 HTTPS:} Hyper Text Transfer Protocol Secure
    \item \textbf{3.1 UI:} User Interface
    \item \textbf{3.1 UX:} User Experience
\end{itemize}
\section{Stakeholder Requirements Specification}
After talking the client, Phil Sadow, he expressed hesitations regarding specific stakeholder requirements. He mentioned the value in having owners of blacklisted Teslas (Tesla cars that have been totaled and are no longer supported by the company) to once again be able to enjoy and service their vehicles. Further, Sadow has placed the value of the Tesla app as the main requirement and evaluation of the ultimate project. If the app has the same responsiveness and usability as the application, Sadow believes this will be roughly all that is necessary for users and for himself.\\
Further studies can and will be conducted looking into what users of these Teslas will be wanting out of an application and further stakeholder requirements can be created through the process of this research. The research will generally take place during development of the back-end and hopefully finish by the time front-end development has begun. 

\section{System Requirements Specification}
\subsection{System Purpose}
This application serves to provide controls in a web app for salvaged Teslas since these cars do not get first-party support anymore.
\subsection{System Scope}
This web application will offer the same functionality as Tesla's first party app. 
This includes but is not limited to starting the car, summoning the car to the user's location, and accessing the climate controls.

\subsection{System Overview}
\subsubsection{System Context}
This web app will allow its users to control their Tesla vehicle. The user will log into their specific car and have access to different actions they can do. Clicking one of these will cause the car to do the selected action, i.e. turn on or off.
\subsubsection{System Functions}
\begin{itemize}
    \item Lock and unlock the car
    \item Check charging process of car and start or stop it
    \item Locate the car wherever it is
    \item Control the climate system in the car
    \item Control the panoramic roof
    \item Summon vehicle out of garage or parking space
\end{itemize}
\subsubsection{User Characteristics}
The user will be someone who owns a salvaged Tesla and wishes to use all the features that someone who has a non-salvaged Tesla would have.
\subsubsection{System Interfaces}
The system will be accessed through any web browser to control the user's car.
\subsection{System Operations}
\subsubsection{Reliability}
There should be as little downtime as possible when working because any downtime means the user's can't control their cars.
It's crucial to make this system as dependable as possible since as it scales up, more and more people will be using it so any down time would cause large problems.
\subsection{System Modes and States}
The system will exist in a web app.
\subsection{Physical Characteristics}
\subsubsection{System Security}
The site will have specific logins for each car.
Once logged in, all traffic will be encrypted with HTTPS and will be sent to the car over an OpenVPN connection.
\subsection{Verification}
The application will be completed when its functionality matches that of Tesla's own app.
\subsection{Assumptions and Dependencies}
We assume that the user will be able to connect to the internet since this application runs in a web browser and needs to connect to a remote database.

\section{Software Requirements Specification}

\subsection{Purpose}
The general purpose of this web application is to establish a front-end interface which the user can use as a control panel for their car, and a back-end that will connect this interface to a database of cars, and the car's API\cite{1}.

\subsection{Scope}
To create this interface, we will implement a web page that displays all of the required information and provides vehicle controls that the user can interact with. 
We will make the web-page dynamic and interactive by making use of page styling and JavaScript.
The objective with this web page is to make it a slick user experience similar to that offered by the official Tesla app. 
Some specific goals include dynamic page updates, and use of graphics to display vehicle status.\\*
We will also write a back-end script for the web app.
The primary function of this will be to take requests from the web page and either pull data from the database of salvaged vehicles or pass it on to the car via the vehicle API.

\subsection{Product Perspective}
This web app will have to interact with the salvaged Teslas that it is controlling. It will do this using the Tesla API, which allows us to submit commands and fetch state information to and from the car via JSON objects. It will connect to the car using a VPN.

\subsection{Product Functions}
The web application will have the same functions similar to those described in 3.3.2 System Functions. In the event that a native iOS or Android application is completed, these functions will again remain the same, even with a different UI. 

\subsection{Performance Metrics}

    There are two main metrics that will be examined within the development of the web app. 
    The first, which is difficult to measure, is to have comparable usability to the official Tesla app.
    As laid out by the client Phil Sadow, the project's purpose is to replace the Tesla app, and it should at least have all of the abilities of the official Tesla app.
    Our application will display more diagnostic information than the Tesla app, but the comparison between the applications is the main metric for completion.
    
    The second metric we will be using is the speed and responsiveness of the application to requests by users.
    For example, when the summon button is clicked, we do not want the request to take more than a couple of seconds to be sent out and for the summon action to finally happen.
    It should be expected that the application should at least respond to a user's request within a second, and for the request to be initiated by the vehicle within two to three seconds.
    While the speed of an internet connection can affect the how quickly these actions take place, we have to assume there will be a sufficient connection and that the application will not be suffering from a heavy workload.

\section{Information Item Content} %not really typing. more just getting the layout (bibliography, chart, and stuff)

%Burt's notes - Back-end planning: ~1-2 weeks, UX Research: 3 weeks (can be during other development), Planning for booth

% Fall Term | Winter Term | Spring Term
% fall term - all documenting and designing (architecture - languages- frameworks
% winter term - backend first, frontend next testing? break down front and and back end'
% spring term - any remaining dev (should be small hopefully) testing and designing booth

%---Gantt Chart Goes Here---%
\begin{ganttchart}{1}{30}
\gantttitle{Tesla Web Application}{30} \\
\gantttitle{Fall Term}{10}
\gantttitle{Winter Term}{10}
\gantttitle{Spring Term}{10} \\
\gantttitlelist{1,...,30}{1} \\
\ganttbar{Design}{1}{10} \\
\ganttbar{Re-write Backend}{11}{13} \\
\ganttlinkedbar{API Integration}{14}{16} \\
\ganttlinkedbar{Backend Gives Diagnostics}{17}{20} \\
\ganttlinkedbar{API Controls Car}{21}{25} \\
\ganttbar{UX Research}{11}{14} \\
\ganttlinkedbar{UX Design}{15}{16} \\
\ganttlinkedbar{Front End Template}{16}{17} \\ 
\ganttlinkedbar{Componetize Front End}{18}{20} \\
\ganttbar{Testing}{23}{30} \\
\ganttbar{Design Booth}{27}{28} \\
\ganttlinkedbar{Create Booth}{29}{30}
\ganttlink{elem4}{elem9}
\ganttlink{elem8}{elem9}
\end{ganttchart}

\bibliographystyle{IEEEtran}
\bibliography{references}
\end{document}